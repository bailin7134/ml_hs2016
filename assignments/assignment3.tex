% --------------------------------------------------------------
% This is all preamble stuff that you don't have to worry about.
% Head down to where it says "Start here"
% --------------------------------------------------------------

\documentclass[12pt]{article}

\usepackage[margin=1in]{geometry} 
\usepackage{amsmath,amsthm,amssymb}
\usepackage{color}

\makeatletter

\renewcommand\section{\@startsection {section}{1}{\z@}%
	{-3.5ex \@plus -1ex \@minus -.2ex}%
	{2.3ex \@plus.2ex}%
	{\normalfont\large\bfseries}}% from \Large
\renewcommand\subsection{\@startsection{subsection}{2}{\z@}%
	{-3.25ex\@plus -1ex \@minus -.2ex}%
	{1.5ex \@plus .2ex}%
	{\normalfont\large\bfseries}}% from \large
\makeatother

\begin{document}
	
	% --------------------------------------------------------------
	%                         Start here
	% --------------------------------------------------------------
	
	%\renewcommand{\qedsymbol}{\filledbox}
	
	\title{\textbf{Machine Learning Assignment \#3}\\
	Universit{\"a}t Bern}%replace X with the appropriate number
	\author{Lin Bai 09935404} %replace with your name
	
	\maketitle
	
	%%%%%%%%%%%%%%%%%%%%%%%%%%%%%%%%%%%%%%%%%%%%%%%%%%%%%%%%%%%%%%%%%%%%%%%%%%%%%%%%%%%%%%%%
	%%%%%%   notations
	%%%%%%%%%%%%%%%%%%%%%%%%%%%%%%%%%%%%%%%%%%%%%%%%%%%%%%%%%%%%%%%%%%%%%%%%%%%%%%%%%%%%%%%%
%	\section*{notations}
%	$A, B, C\in \mathbb{R}^{n\times n}$ are $n\times n$ matrices, $x, a, b\in \mathbb{R}^n$ are column vectors.

	%%%%%%%%%%%%%%%%%%%%%%%%%%%%%%%%%%%%%%%%%%%%%%%%%%%%%%%%%%%%%%%%%%%%%%%%%%%%%%%%%%%%%%%%
	%%%%%%   question 1, sub question (a)
	%%%%%%%%%%%%%%%%%%%%%%%%%%%%%%%%%%%%%%%%%%%%%%%%%%%%%%%%%%%%%%%%%%%%%%%%%%%%%%%%%%%%%%%%
	\section*{Solution of question 1}
	According to the definition of mutlivariate normal distribution, the probability density function is
	\begin{equation}{\label{mul_gaus}}
		p(x;\mu,\varSigma)=\frac{1}{(2\pi)^{n/2}\mid\varSigma\mid^{1/2}}exp\left(\frac{1}{2}(x-\mu)^T\varSigma^{-1}(x-\mu)\right)
	\end{equation}
	Suppose 
	\begin{equation}
		\left[
		\begin{array}{c}  
			x_A\\
			x_B
		\end{array}
		\right]
		\sim\mathcal{N}
		\left(
			\left[
			\begin{array}{c}
				\mu_A\\
				\mu_B
			\end{array}
			\right],
			\left[
			\begin{array}{cc}
				\varSigma_{AA} & \varSigma_{AB}\\
				\varSigma_{BA} & \varSigma_{BB}
			\end{array}
			\right]
		\right)		
	\end{equation}
	, where $x_A\in \mathbb{R}^n$, $x_B\in \mathbb{R}^n$. The marginal densities,
	\begin{equation}
		p(x_A)=\int_{x_B\in \mathbb{R}^n}p(x_A,x_B;\mu,\varSigma)dx_B
	\end{equation}
	\begin{equation}
		p(x_B)=\int_{x_A\in \mathbb{R}^n}p(x_A,x_B;\mu,\varSigma)dx_A
	\end{equation}
	are Gaussian
	\begin{equation}
		x_A\sim\mathcal{N}(\mu_A,\varSigma_{AA})
	\end{equation}
	\begin{equation}
		x_B\sim\mathcal{N}(\mu_B,\varSigma_{BB})
	\end{equation}
	\noindent
	Apply the equations above to equation \ref{mul_gaus}, then
	\begin{equation}
	\begin{aligned}
	\begin{split}
		p(x_A)&=\int_{x_B\in \mathbb{R}^n}p(x_A,x_B;\mu,\varSigma)dx_B\\
		&=\frac{1}{(2\pi)^n\mid
		\begin{array}{cc}
			\varSigma_{AA} & \varSigma_{AB}\\
			\varSigma_{BA} & \varSigma_{BB}
		\end{array}
		\mid^{1/2}
		}\\&
		\int_{x_B\in \mathbb{R}^n}exp(
		-\frac{1}{2}
		\left[
		\begin{array}{c}
			x_A-\mu_A\\
			x_B-\mu_B
		\end{array}
		\right]^T
		\left[
		\begin{array}{cc}
			\varSigma_{AA} & \varSigma_{AB}\\
			\varSigma_{BA} & \varSigma_{BB}
		\end{array}
		\right]^{-1}
		\left[
		\begin{array}{c}
			x_A-\mu_A\\
			x_B-\mu_B
		\end{array}
		\right]
		)dx_B
	\end{split}
	\end{aligned}
	\end{equation}
	\noindent
	Define
	\begin{equation}
		V=\left[
		\begin{array}{cc}
			\varSigma_{AA} & \varSigma_{AB}\\
			\varSigma_{BA} & \varSigma_{BB}
		\end{array}
		\right]=\varSigma^{-1}
	\end{equation}
	\noindent
	The marginal density is
	\begin{equation}
	\begin{aligned}
	\begin{split}
	p(x_A)&=\frac{1}{Z}
	\int_{x_B\in \mathbb{R}^n}exp(
	-\frac{1}{2}
	\left[
	\begin{array}{c}
	x_A-\mu_A\\
	x_B-\mu_B
	\end{array}
	\right]^T
	\left[
	\begin{array}{cc}
	\varSigma_{AA} & \varSigma_{AB}\\
	\varSigma_{BA} & \varSigma_{BB}
	\end{array}
	\right]^{-1}
	\left[
	\begin{array}{c}
	x_A-\mu_A\\
	x_B-\mu_B
	\end{array}
	\right]
	)dx_B\\
	&=\frac{1}{Z}exp\left(-\left[\frac{1}{2}(x_A-\mu_A)^TV_{AA}(x_A-\mu_A)+\frac{1}{2}(x_A-\mu_A)^TV_{AB}(x_B-\mu_B)+\right.\right.\\
	&\left.\left.\frac{1}{2}(x_B-\mu_B)^TV_{BA}(x_A-\mu_A)+\frac{1}{2}(x_B-\mu_B)^TV_{BB}(x_B-\mu_B)]\right)\right)dx_B
	\end{split}
	\end{aligned}
	\end{equation}
	
	
	
	
	% --------------------------------------------------------------
	%     You don't have to mess with anything below this line.
	% --------------------------------------------------------------
	
\end{document}