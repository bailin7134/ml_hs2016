% --------------------------------------------------------------
% This is all preamble stuff that you don't have to worry about.
% Head down to where it says "Start here"
% --------------------------------------------------------------
 
\documentclass[12pt]{article}
 
\usepackage[margin=1in]{geometry} 
\usepackage{amsmath,amsthm,amssymb}
\usepackage{color}

\makeatletter
\renewcommand\section{\@startsection {section}{1}{\z@}%
	{-3.5ex \@plus -1ex \@minus -.2ex}%
	{2.3ex \@plus.2ex}%
	{\normalfont\large\bfseries}}% from \Large
\renewcommand\subsection{\@startsection{subsection}{2}{\z@}%
	{-3.25ex\@plus -1ex \@minus -.2ex}%
	{1.5ex \@plus .2ex}%
	{\normalfont\large\bfseries}}% from \large
\makeatother
 
\begin{document}
 
% --------------------------------------------------------------
%                         Start here
% --------------------------------------------------------------
 
%\renewcommand{\qedsymbol}{\filledbox}
 
\title{\textbf{Machine Learning Assignment \#2}}%replace X with the appropriate number
\author{Lin Bai 09935404} %replace with your name
 
\maketitle

%%%%%%%%%%%%%%%%%%%%%%%%%%%%%%%%%%%%%%%%%%%%%%%%%%%%%%%%%%%%%%%%%%%%%%%%%%%%%%%%%%%%%%%%
%%%%%%   question 1, sub question (a)
%%%%%%%%%%%%%%%%%%%%%%%%%%%%%%%%%%%%%%%%%%%%%%%%%%%%%%%%%%%%%%%%%%%%%%%%%%%%%%%%%%%%%%%%
\section*{Solution of question 1}
\subsection*{(a)}
Since both $a$ and $x$ are column vector with n elements.\\
we know $a^{T}x = (a_1, a_2, ... a_n)$
\begin{equation}\label{a_time_x}
	a^{T}x=  
		\left(  
		\begin{array}{cccc}  
			a_{1}, &  
			a_{2}, &   
			\cdots &  
			a_{n}  
		\end{array}  
		\right)  
		\left(  
		\begin{array}{c}  
			x_{1} \\   
			x_{2} \\  
			\vdots \\  
			x_{n}  
		\end{array}  
		\right)
		=a_1x_1+a_2x_2+\cdots a_nx_n
		= \sum_{i=1}^n a_i x_i
\end{equation}
according to the mathmatics definition of $\nabla_xf(x)$ and equation (\ref{a_time_x}), we know
\begin{equation}\label{gradient_atx}
	\nabla_x(a^Tx)=
		\left(  
		\begin{array}{c}  
			\frac{\partial(a^Tx)}{\partial x_1} \\
			\frac{\partial(a^Tx)}{\partial x_2} \\
			\vdots \\
			\frac{\partial(a^Tx)}{\partial x_n} 
		\end{array}
		\right)
		=
		\left(  
		\begin{array}{c}  
			\frac{\partial(\sum_{i=1}^n a_i x_i)}{\partial x_1} \\
			\frac{\partial(\sum_{i=1}^n a_i x_i)}{\partial x_2} \\
			\vdots \\
			\frac{\partial(\sum_{i=1}^n a_i x_i)}{\partial x_n} 
		\end{array}
		\right)
\end{equation}
due to
\begin{equation}\label{gradient_atx_i}
	\frac{\partial(\sum_{i=1}^n a_i x_i)}{\partial x_i}=a_i
\end{equation}
we know the equation (\ref{gradient_atx}) will become
\begin{equation}\label{gradient_atx_final}
	\nabla_x(a^Tx)=
	\left(  
		\begin{array}{c}  
			a_1 \\
			a_2 \\
			\vdots \\
			a_n 
		\end{array}
	\right)
	= a
\end{equation}
therefore, $$\nabla_x(a^Tx)=a$$

%%%%%%%%%%%%%%%%%%%%%%%%%%%%%%%%%%%%%%%%%%%%%%%%%%%%%%%%%%%%%%%%%%%%%%%%%%%%%%%%%%%%%%%%
%%%%%%%   question 1, sub question (b)
%%%%%%%%%%%%%%%%%%%%%%%%%%%%%%%%%%%%%%%%%%%%%%%%%%%%%%%%%%%%%%%%%%%%%%%%%%%%%%%%%%%%%%%%
\subsection*{(b)}
{\color{red} ???????????????????????????????}\\
The $i$th element of $x^TA$ is
\begin{equation}\label{xt_a}
	[x^TA]_{i}=
	\left( 
	\left(  
	\begin{array}{c}  
		x_1 \\
		x_2 \\
		\vdots \\
		x_n 
	\end{array}
	\right)
	\left(  
	\begin{array}{c}  
		a_{11}, a_{12},\cdots a_{1n} \\
		a_{21}, a_{22},\cdots a_{2n} \\
		\cdots \\
		a_{n1}, a_{n2},\cdots a_{nn} \\
	\end{array}  
	\right)  
	\right)_{i}\\
	=x_1a_{i1} + x_1a_{i1} + \cdots x_na_{in}
	=\sum_{p=1}^n x_pa_{ip} 
\end{equation}
same princple, $x^TAx$'s product is
\begin{equation}\label{xt_a_x}
	x^TAx = (\sum_{p=1}^n x_pa_{1p})x_1 + (\sum_{p=1}^n x_pa_{2p})x_2 + \cdots (\sum_{p=1}^n x_pa_{np})x_n
	= \sum_{p=1}^n \sum_{q=1}^n x_pa_{pq}x_q
\end{equation}
therefore the gradient of $x^TAx$ is
\begin{equation}\label{gradient_xtax}
	\nabla_x(x^TAx)=
	\left(  
	\begin{array}{c}  
		\frac{\partial(\sum_{p=1}^n \sum_{q=1}^n x_pa_{pq}x_q)}{\partial x_1} \\
		\frac{\partial(\sum_{p=1}^n \sum_{q=1}^n x_pa_{pq}x_q)}{\partial x_2} \\
		\vdots \\
		\frac{\partial(\sum_{p=1}^n \sum_{q=1}^n x_pa_{pq}x_qs)}{\partial x_n} 
	\end{array}
	\right)
\end{equation}
take question (\ref{gradient_xtax})'s $i$th element for example, the $\frac{\partial(\sum_{p=1}^n \sum_{q=1}^n x_pa_{pq}x_qs)}{\partial x_n}$ contains one 2-order expression of $x_i$, that is ${x_i}^2a_{ii}$; and two 1-order expressions, they are $\sum_{p=1}^n x_pa_{pi}$ and $\sum_{q=1}^n x_qa_{iq}$, but both $q$ and $p$ cannot equals $i$.
\begin{equation}\label{gradient_xtax_i}
	\frac{\partial(\sum_{p=1}^n \sum_{q=1}^n x_pa_{pq}x_q)}{\partial x_i} \\
	=\frac{\partial(\sum_{p=1}^n x_ix_pa_{pi}+\sum_{q=1}^n x_ix_qa_{iq}+x_ix_ia_{ii})}{\partial x_i}
\end{equation}
, where $q\neq i$ and $p\neq i$.
\\
since matrix $A$ is symmetrical, $a_{ij=a_{ji}}$, equation (\ref{gradient_xtax_i}) becomes
\begin{equation}\label{gradient_xtax_i_mod}
	\frac{\partial(\sum_{p=1}^n \sum_{q=1}^n x_pa_{pq}x_q)}{\partial x_i} \\
	=\frac{\partial(2\sum_{p=1}^n x_ix_pa_{pi}}{\partial x_i}=2\sum_{p=1}^n x_ix_pa_{pi}+2x_ia_{ii}
\end{equation}
, where $p\neq i$.\\
Since $2x_ia_{ii}$ is the expression of $x_ix_pa_{pi}$ when $p=i$, we know

\begin{equation}\label{gradient_xtax_i_final}
	\frac{\partial(\sum_{p=1}^n \sum_{q=1}^n x_pa_{pq}x_q)}{\partial x_i} \\
	=2\sum_{p=1}^n xpa_{ip}
\end{equation}
so, the $i$th element of $\nabla_x(x^TAx)$ is $2\sum_{p=1}^n x_pa_{ip}$.\\
\\
\noindent
While $2x^TA$'s $i$th element is $2\sum_{p=1}^n x_pa_{ip}$ as well.\\
Therefore, $\nabla_x(x^TAx)=2x^TA$ is proved.

%%%%%%%%%%%%%%%%%%%%%%%%%%%%%%%%%%%%%%%%%%%%%%%%%%%%%%%%%%%%%%%%%%%%%%%%%%%%%%%%%%%%%%%%
%%%%%%   question 2
%%%%%%%%%%%%%%%%%%%%%%%%%%%%%%%%%%%%%%%%%%%%%%%%%%%%%%%%%%%%%%%%%%%%%%%%%%%%%%%%%%%%%%%%
\section*{Solution of question 2}
{\color{red} ???????????????????????????????}\\
According to the chain rule
\begin{equation}\label{chain_rule}
\nabla_x(u^Tu)=\frac{\partial(u^Tu)}{\partial x}=\frac{\partial(u^T)}{\partial x}u+u^T\frac{\partial u}{\partial x}=(\frac{\partial u}{\partial x})^Tu+u^T\frac{\partial u}{\partial x}=2u^T\frac{\partial u}{\partial x}
\end{equation}
in our case $u=Ax-b$
\begin{equation}\label{Ax-b_grad}
\nabla_x\lVert Ax-b\rVert_2^2=\nabla_x (Ax-b)^TAx-b=2(Ax-b)^T\frac{\partial(Ax-b)}{\partial x}
\end{equation}
since vector $b$ is constant for $x$, it can be ignored. from the conclusion of question 1 (a).\\
we know
\begin{equation}\label{Ax-b_grad_final}
\nabla_x\lVert Ax-b\rVert_2^2=\nabla_x (Ax-b)^TAx-b=2(Ax-b)^T\frac{\partial(Ax-b)}{\partial x}=2(Ax-b)^T
\left(  
\begin{array}{c}  
	a_{11} \\
	a_{22} \\
	\vdots \\
	a_{nn} 
\end{array}
\right)
\end{equation}
where $a_{ii}$ is $[A]_{(i,i)}$.

%%%%%%%%%%%%%%%%%%%%%%%%%%%%%%%%%%%%%%%%%%%%%%%%%%%%%%%%%%%%%%%%%%%%%%%%%%%%%%%%%%%%%%%%
%%%%%%   question 3, sub question (a)
%%%%%%%%%%%%%%%%%%%%%%%%%%%%%%%%%%%%%%%%%%%%%%%%%%%%%%%%%%%%%%%%%%%%%%%%%%%%%%%%%%%%%%%%
\section*{Solution of question 3}
\subsection*{(a)}
the trace of matrix $A$ is $\sum_{i=1}^n a_{ii}$.\\
so the gradient is $tr(A)$ is
\begin{equation}\label{trA_grad}
	\nabla_A(tr(A))=
	\left(  
	\begin{array}{cccc}  
		\frac{\partial\sum_{i=1}^n a_{ii}}{\partial a_{11}} & \frac{\partial\sum_{i=1}^n a_{ii}}{\partial a_{12}} \cdots \frac{\partial\sum_{i=1}^n a_{ii}}{\partial a_{1n}}\\
		\frac{\partial\sum_{i=1}^n a_{ii}}{\partial a_{21}} & \frac{\partial\sum_{i=1}^n a_{ii}}{\partial a_{22}} \cdots \frac{\partial\sum_{i=1}^n a_{ii}}{\partial a_{1n}}\\
		\vdots \\
		\frac{\partial\sum_{i=1}^n a_{ii}}{\partial a_{n1}} & \frac{\partial\sum_{i=1}^n a_{ii}}{\partial a_{n2}} \cdots \frac{\partial\sum_{i=1}^n a_{ii}}{\partial a_{nn}}\\
	\end{array}
	\right)
	=
	\left(  
	\begin{array}{cccc}
		1 & 0 & \cdots & 0\\
		0 & 1 & \cdots & 0\\
		\vdots\\
		0 & 0 & \cdots & 1\\
	\end{array}
	\right)
	= I
\end{equation}

%%%%%%%%%%%%%%%%%%%%%%%%%%%%%%%%%%%%%%%%%%%%%%%%%%%%%%%%%%%%%%%%%%%%%%%%%%%%%%%%%%%%%%%%
%%%%%%   question 3, sub question (b)
%%%%%%%%%%%%%%%%%%%%%%%%%%%%%%%%%%%%%%%%%%%%%%%%%%%%%%%%%%%%%%%%%%%%%%%%%%%%%%%%%%%%%%%%
\subsection*{(b)}
The trace of product $AB$ is
\begin{equation}\label{ab_ii}
	tr(AB)=\sum_{i=1}^n (AB)_{(i,i)}=\sum_{i=1}^n\sum_{j=1}^n a_{ij}b_{ji}
\end{equation}
so the gradient of $AB$ respect to $A$ is
\begin{equation}\label{trAB_grad}
\begin{split}
	\nabla_A(tr(AB))=&
	\left(  
	\begin{array}{cccc}
		\frac{\partial\sum_{i=1}^n\sum_{j=1}^n a_{ij}b_{ji}}{\partial a_{11}} & \frac{\partial\sum_{i=1}^n\sum_{j=1}^n a_{ij}b_{ji}}{\partial a_{12}} \cdots \frac{\partial\sum_{i=1}^n\sum_{j=1}^n a_{ij}b_{ji}}{\partial a_{1n}}\\
		\frac{\partial\sum_{i=1}^n\sum_{j=1}^n a_{ij}b_{ji}}{\partial a_{21}} & \frac{\partial\sum_{i=1}^n\sum_{j=1}^n a_{ij}b_{ji}}{\partial a_{22}} \cdots \frac{\partial\sum_{i=1}^n\sum_{j=1}^n a_{ij}b_{ji}}{\partial a_{1n}}\\
		\vdots \\
		\frac{\partial\sum_{i=1}^n\sum_{j=1}^n a_{ij}b_{ji}}{\partial a_{n1}} & \frac{\partial\sum_{i=1}^n\sum_{j=1}^n a_{ij}b_{ji}}{\partial a_{n2}} \cdots \frac{\partial\sum_{i=1}^n\sum_{j=1}^n a_{ij}b_{ji}}{\partial a_{nn}}\\	
	\end{array}
	\right)\\
	&=
	\left(
	\begin{array}{cccc}	
		b_{11} & b_{21} & \cdots & b_{n1}\\
		b_{12} & b_{22} & \cdots & b_{n2}\\
		\vdots \\
		b_{1n} & b_{2n} & \cdots & b_{nn}\\
	\end{array}
	\right)
	=B^T
\end{split}
\end{equation}

%%%%%%%%%%%%%%%%%%%%%%%%%%%%%%%%%%%%%%%%%%%%%%%%%%%%%%%%%%%%%%%%%%%%%%%%%%%%%%%%%%%%%%%%
%%%%%%   question 3, sub question (c)
%%%%%%%%%%%%%%%%%%%%%%%%%%%%%%%%%%%%%%%%%%%%%%%%%%%%%%%%%%%%%%%%%%%%%%%%%%%%%%%%%%%%%%%%
\subsection*{(c)}
from the conclusion from (b), we know
\begin{equation}\label{aa_ii}
	tr(A^TA)=\sum_{i=1}^n (A^TA)_{(i,i)}=\sum_{i=1}^n\sum_{j=1}^n a_{ij}^2
\end{equation}
so the gradient of $A^TA$ respect to $A$ is
\begin{equation}\label{trAA_grad}
\begin{split}
	\nabla_A(tr(AB))=&
	\left(  
	\begin{array}{cccc}
		\frac{\partial\sum_{i=1}^n\sum_{j=1}^n a_{ij}^2}{\partial a_{11}} & \frac{\partial\sum_{i=1}^n\sum_{j=1}^n a_{ij}^2}{\partial a_{12}} \cdots \frac{\partial\sum_{i=1}^n\sum_{j=1}^n a_{ij}^2}{\partial a_{1n}}\\
		\frac{\partial\sum_{i=1}^n\sum_{j=1}^n a_{ij}^2}{\partial a_{21}} & \frac{\partial\sum_{i=1}^n\sum_{j=1}^n a_{ij}^2}{\partial a_{22}} \cdots \frac{\partial\sum_{i=1}^n\sum_{j=1}^n a_{ij}^2}{\partial a_{2n}}\\
		\vdots \\
		\frac{\partial\sum_{i=1}^n\sum_{j=1}^n a_{ij}^2}{\partial a_{n1}} & \frac{\partial\sum_{i=1}^n\sum_{j=1}^n a_{ij}^2}{\partial a_{n2}} \cdots \frac{\partial\sum_{i=1}^n\sum_{j=1}^n a_{ij}^2}{\partial a_{nn}}\\	
	\end{array}
	\right)\\
	&=
	\left(  
	\begin{array}{cccc}
		\frac{\partial a_{11}^2}{\partial a_{11}} & \frac{\partial a_{12}^2}{\partial a_{12}} \cdots \frac{\partial a_{1n}^2}{\partial a_{1n}}\\
		\frac{\partial a_{21}^2}{\partial a_{21}} & \frac{\partial a_{22}^2}{\partial a_{22}} \cdots \frac{\partial a_{2n}^2}{\partial a_{2n}}\\
		\vdots \\
		\frac{\partial a_{n1}^2}{\partial a_{n1}} & \frac{\partial a_{n2}^2}{\partial a_{n2}} \cdots \frac{\partial a_{nn}^2}{\partial a_{nn}}\\	
	\end{array}
	\right)\\
	&=
	\left(
	\begin{array}{cccc}	
		2a_{11} & 2a_{12} & \cdots & 2a_{1n}\\
		2a_{21} & 2a_{22} & \cdots & 2a_{2n}\\
		\vdots \\
		2a_{n1} & 2a_{n2} & \cdots & 2a_{nn}\\
	\end{array}
	\right)
	=2A
\end{split}
\end{equation}

%%%%%%%%%%%%%%%%%%%%%%%%%%%%%%%%%%%%%%%%%%%%%%%%%%%%%%%%%%%%%%%%%%%%%%%%%%%%%%%%%%%%%%%%
%%%%%%   question 4, sub question (a)
%%%%%%%%%%%%%%%%%%%%%%%%%%%%%%%%%%%%%%%%%%%%%%%%%%%%%%%%%%%%%%%%%%%%%%%%%%%%%%%%%%%%%%%%
\subsection*{Solution of question 4}
\subsection*{(a)}
the 4 matrix norm properties are:
\begin{enumerate}
	\item For all $x\in \mathbb{R}^{n}$, $f(x)\geq0$ (non-negativity)
	\item $f(x)=0$ if and only if $x=0$ (definiteness)
	\item For all $x\in \mathbb{R}^{n}$, $t\in \mathbb{R}$, $f(tx)=\mid t\mid f(x)$ (homogeneity)
	\item For all $x, y\in \mathbb{R}^{n}$, $f(x+y)\le f(x)+f(y)$ (triangle inequality)
\end{enumerate}
The definition of Frobenius norm is
\begin{equation}\label{frob_norm}
	\lVert A\rVert_F=\sqrt{\sum_{i=1}^m\sum_{j=1}^nA_{ij}^2}=\sqrt{tr(A^TA)}
\end{equation}
\textbf{non-negativity}\\
since matrix is in real number space, $A_{ij}^2\geq 0$, so sum of each $A_{ij}^2$ must be greater or equal to 0, so as it's square root. So $\lVert A\rVert_F\geq 0$\\
\textbf{definiteness}\\
if $\sqrt{\sum_{i=1}^m\sum_{j=1}^nA_{ij}^2}=0$, $\sum_{i=1}^m\sum_{j=1}^nA_{ij}^2$ must be 0. And $A_{ij}$ msut be 0. Thus, the matrix $A$ is null matrix.\\
So the conclusion, if and only if matrix $A$ is 0, it's Frobenius norm is 0.\\
\textbf{homogeneity}\\
The Frobenius norm of $tA$ (where $t$ is a constant) is 
\begin{equation}\label{t_frob_norm}
	\lVert tA\rVert_F=\sqrt{\sum_{i=1}^m\sum_{j=1}^n(tA_{ij})^2}=\mid t\mid\sqrt{tr(A^TA)}=\mid t\mid\lVert A\rVert_F
\end{equation}
So it fulfill the property of homogeneity.\\
\textbf{triangle inequality}\\
\begin{equation}\label{ab_frob_norm}
	\lVert A+B\rVert_F=\sqrt{\sum_{i=1}^m\sum_{j=1}^n(A_{ij}+B_{ij})^2}
\end{equation}
Extract one arbitrary element for example, $\sqrt{(A_{ij}+B_{ij})^2}=\mid A_{ij}+B_{ij}\mid$, and it's not hard to get 
\begin{equation}\label{equa}
	\mid A_{ij}+B_{ij}\mid \le \mid A_{ij}\mid +\mid B_{ij}\mid
\end{equation}
so we could get the conclusion 
\begin{equation}\label{equa2}
	\sqrt{\sum_{i=1}^m\sum_{j=1}^n(A_{ij}+B_{ij})^2} \le \sqrt{\sum_{i=1}^m\sum_{j=1}^n A_{ij}^2} + \sqrt{\sum_{i=1}^m\sum_{j=1}^n B_{ij}^2}
\end{equation}
\begin{equation}\label{equa_final}
	\lVert A+B\rVert_F\le \lVert A\rVert_F+\lVert B\rVert_F
\end{equation}
Therefore, Frobenius norm fuifull the property of triangle inequality as well.

%%%%%%%%%%%%%%%%%%%%%%%%%%%%%%%%%%%%%%%%%%%%%%%%%%%%%%%%%%%%%%%%%%%%%%%%%%%%%%%%%%%%%%%%
%%%%%%   question 4, sub question (b)
%%%%%%%%%%%%%%%%%%%%%%%%%%%%%%%%%%%%%%%%%%%%%%%%%%%%%%%%%%%%%%%%%%%%%%%%%%%%%%%%%%%%%%%%
\subsection*{(b)}
due to the fact $\lVert A+B\rVert_F=tr(A^TA)$, we could have
\begin{equation}\label{trAA_from}
	\nabla_A(\lVert A\rVert_F^2)=\nabla_A(tr(A^TA))
\end{equation}
then this question is the same as question 3(c), from equation and (\ref{aa_ii}) and (\ref{trAA_grad})
we can get
\begin{equation}\label{trAA_from}
	\nabla_A(\lVert A\rVert_F^2)=2A
\end{equation}
	



% --------------------------------------------------------------
%     You don't have to mess with anything below this line.
% --------------------------------------------------------------
 
\end{document}